\documentclass{beamer}
\usepackage{hyperref}
\usenavigationsymbolstemplate{}
\setbeamertemplate{footline}[frame number]
\newcounter{saveenumi}
\newcommand{\seti}{\setcounter{saveenumi}{\value{enumi}}}
\newcommand{\conti}{\setcounter{enumi}{\value{saveenumi}}}
\begin{document}
\everymath{\displaystyle}
\begin{frame}
\frametitle{An Event Driven Algorithm for Simulating the Decision Making
Model}
\begin{itemize}
\item An event driven algorithm only does computations when something happens.
In the DMM, this corresponds to a unit changing state.

\item The idea here is to draw a time at which a unit will switch state, and
modify this time to account for the dynamics of the other units.
\end{itemize}
\begin{equation}
\frac{\partial p_1}{\partial t}=-g_0 e^{\frac{K}{2d}(N_2-N_1)}p_1
\end{equation}
\begin{equation}
p_1=e^{-tg_0 e^{\frac{K}{2d}(N_2-N_1)}}
\end{equation}
Where $p_1$ is the probability of being in state $1$ given that the unit is initially in state $1$ and no transitions have occurred so far, $d$ is the number of neighbours divided by $2$, and $N_i$ is the number of neighbours in the state $i$.  Interpreting this as a survival probability, one can draw a random
number from [0,1] to obtain an estimate (exact in the case that none of
the neighbours change state) for the time at which the unit will next
change state.
\end{frame}
\begin{frame}
Suppose the random number drawn is $r$, then one can find the
corresponding time:
\begin{equation}
r=e^{-tg_0 e^{\frac{K}{2d}(N_2-N_1)}}\quad\Rightarrow\quad t=-\frac{e^
{-\frac{K}{2d}(N_2-N_1)}}{g_0}\ln(r).
\end{equation}
Since there is no timestep in this algorithm, one is free to choose
dimensionless units, yielding:
\begin{equation}\label{tdist}
t=-e^{-\frac{K}{2d}(N_2-N_1)}\ln(r).
\end{equation}
\begin{enumerate}
\item Initially, one assigns the states of the units, then draws a
random number for each, and determines the estimated switching time for
each unit via Eq.~(\ref{tdist}).
\item Find the unit with the smallest time, draw a new time for that
unit according to Eq.~(\ref{tdist}), and update the times of its
neighbours according to Eq.~(\ref{updt}).  This can be done in a
computational time proportional to $d$, and independent of the total
number of units \cite{O1BPQ}.
\seti
\end{enumerate}
\end{frame}
\begin{frame}
%\begin{enumerate}
%\conti

%\seti
%\end{enumerate}
\begin{align}\label{updt}
r=&e^{-t_fe^{\frac{K}{2d}(N_2-N_1)}}e^{-(t_n-t_f)e^{\frac{K}{2d}(N_2-
N_1)}e^{\pm\frac{K}{d}}}\nonumber\\\text{thus, }t_n=&t_f+(t_o-t_f)e^{
\mp\frac{K}{d}}
\end{align}
Where $t_o$ is the old switching time estimate, $t_f$ is the current
time, $t_n$ is the new estimate of the time at which the unit will
switch states, and the sign in the exponent is chosen based on whether
the unit that just switched is now in the same (corresponding to the
lower sign) or opposite (upper sign) state as the neighbour being
updated.
\begin{enumerate}
\conti
\item Repeat step 2 until stopping criterion is satisfied.
\end{enumerate}
The computational time of this algorithm is better than or comparable to
that of the usual approach based on timesteps, but does not suffer from
the possibility of having too large a timestep.
\end{frame}
\begin{frame}
\frametitle{References}
\bibliographystyle{apsrev4-1}
\bibliography{20160429DMMAlg}
\end{frame}
\end{document}
